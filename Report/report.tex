\documentclass[a4paper,12pt]{article}

\usepackage[utf8x]{inputenc}
\usepackage[english]{babel}
\usepackage[margin=1in,includefoot]{geometry}

\usepackage{pgfplots}
\pgfplotsset{width=15cm,compat=1.12}
\usetikzlibrary{arrows}

\usepackage{indentfirst}

\usepackage[backref=false,pagebackref=true,citecolor=blue]{hyperref}
\hypersetup{colorlinks=true,urlcolor=blue,pdfborder={0 0 0}}

\setlength{\parindent}{2em}
\setlength{\parskip}{1em}

%\renewcommand{\arraystretch}{2}

\newcommand{\refspace}{\vspace{-2mm}}
\newcommand{\redarrow}{\textcolor{blue}{$\mathbf{\Rightarrow}$}}
\makeatletter
\def\BR@@bibitem#1#2\par{
	\let\backrefprint\BR@backrefprint
	\def\@linkcolor{black}
	\BRorg@bibitem{#1}#2\redarrow \thinspace \BR@backref{#1}
}
\makeatother

\author{Roland Bogosi}
\title{Market Prediction}

\begin{document}
\thispagestyle{empty}
	 
	\begin{center}
		\vspace{3.1in}
		
		{\huge Market Prediction using Neural Networks}
		
		\vspace{0.4in}
		
		\textbf{\LARGE Assignment Report}
		
		\vspace{0.3in}
		
		{\Large January 4th, 2016}
		
		\vspace{3.2in}
	\end{center}
	
	\begin{flushright}
		{\Large \textit{Roland Bogosi}}
	\end{flushright}

\newpage
\thispagestyle{empty}
\section*{Table of Contents}

	\begingroup
	\renewcommand{\section}[2]{}
	\hypersetup{linkcolor=blue}
	\setlength{\parskip}{0em}
	\tableofcontents
	\endgroup

\newpage
\section{Introduction}

	\tikzstyle{es} = [-triangle 60]
	\begin{figure}[!htbp]
		\centering
		\begin{tikzpicture}[x=1.5cm,y=1.5cm]
			\draw [] (0,0) ellipse (0.5 and 0.5);
			\draw [] (0,1.5) ellipse (0.5 and 0.5);
			\draw [] (0,-1.5) ellipse (0.5 and 0.5);
			\draw [] (2.5,0) ellipse (0.5 and 0.5);
			\draw [] (-2.5,0) ellipse (0.5 and 0.5);
			\node (v1) at (-2,0) {};
			\node (v2) at (-0.5,1.5) {};
			\node (v3) at (-0.5,0) {};
			\node (v4) at (-0.5,-1.5) {};
			\node (v5) at (0.5,1.5) {};
			\node (v7) at (0.5,0) {};
			\node (v8) at (0.5,-1.5) {};
			\node (v6) at (2,0) {};
			\draw [es] (v1) edge (v2);
			\draw [es] (v1) edge (v3);
			\draw [es] (v1) edge (v4);
			\draw [es] (v5) edge (v6);
			\draw [es] (v7) edge (v6);
			\draw [es] (v8) edge (v6);
			\node (v9) at (-4.5,0) {};
			\node (v12) at (4.5,0) {};
			\node (v10) at (-3,0) {};
			\draw [es] (v9) edge (v10);
			\node (v11) at (3,0) {};
			\draw [es] (v11) edge (v12);
			\node at (-2.5,2.5) {\shortstack{Input\\layer}};
			\node at (0,2.5) {\shortstack{Hidden\\layers}};
			\node at (2.5,2.5) {\shortstack{Output\\layer}};
			\node at (0.5,2) {$w_1$};
			\node at (0.5,0.5) {$w_2$};
			\node at (0.5,-1) {$w_3$};
			\node at (3,0.5) {$y_1$};
			\node at (-2,0.5) {$x_1$};
		\end{tikzpicture}
		\caption{Structure of the Neural Network}
		\label{neurnet}
	\end{figure}

	The neural network has a single input, $x_1$, which is the rate of the currency for that day. The output of the network, $y_1$, is the predicted value for the \textit{next} day.
	
	The number of hidden layers in the network varies based on how many days we train and would like to predict. While the general rule is to have $input\_count*2$ hidden layers, I evaluated this during several trial and error runs and it did not seem to work well with this application: even with a high momentum learning parameter, the neural network was not able to accurately fit the data, it only generated an approximate oscillation. I concluded that the number of hidden layers in order to accurately fit the data within $1,000$ iterations, is roughly equal to the number of days taught.

	\begin{figure}[!htbp]
		\centering
		\begin{tikzpicture}
		\begin{axis}[
			xlabel={Days},
			ylabel={Index},
			legend pos=north west,
			ymajorgrids=true,
			grid style=dashed,
			legend entries={Actual Rates,Estimate,Prediction}
		]
			\addplot[blue,mark=*]  table {eur_usd_august_30.txt};
			\addplot[red,mark=*]   table {eur_usd_august_30_neuron.txt};
			\addplot[green,mark=*] table {eur_usd_august_30_neuron_pred.txt};
		\end{axis}
		\end{tikzpicture}
		\caption{EUR/USD Rates for August 2015}
		\label{eurusd2015}
	\end{figure}
	
	\begin{figure}[!htbp]
		\centering
		\begin{tikzpicture}
		\begin{axis}[
			xlabel={Days},
			ylabel={Index},
			legend pos=north west,
			ymajorgrids=true,
			grid style=dashed,
			legend entries={Actual Rates,Estimate,Prediction}
		]
			\addplot[blue,mark=*]  table {eur_usd_august_30.txt};
			\addplot[red,mark=*]   table {eur_usd_august_30_genetic.txt};
			\addplot[green,mark=*] table {eur_usd_august_30_genetic_pred.txt};
		\end{axis}
		\end{tikzpicture}
		\caption{EUR/USD Rates for August 2015}
		\label{eurusd2015}
	\end{figure}
	
	The output of the genetic algorithm, meaning the fittest chromosome, after $1,000$ iterations is the following equation:\\
	$$ (b \cdot \frac{b \cdot \frac{c}{b}}{\frac{c \cdot b}{b}}) \cdot \frac{a}{(b \cdot c) \cdot \frac{\frac{c}{d}}{b}} $$
	
	Which can be simplified to:\\
	$$ \frac{a \cdot b \cdot c}{2} $$
	
\newpage
\section{Bibliography}

	\begingroup
	\renewcommand{\section}[2]{}
		\bibliography{report} 
		\bibliographystyle{ieeetr}
	\endgroup

\end{document}